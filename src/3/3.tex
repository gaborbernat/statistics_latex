\section{Introduction to Probability}

\subsection{The need for probability}

Up to this point we've seen and focused on how to handle data available to us.
Now it's time to see what the data in the real world corresponds in the
theoretical world. The data of the real world, that we usually end up having,
can be viewed as just a sampling of the theoretical world (which has an
infinit number of data points). Even if it really isn't any more data points in
the real world (so we have collected all the existing data points) we can still
pretend that there is and construct a theoretical model around it.

We usually try to draw inferences about our theoretical world by using the data
that we have, which represents the real world. Of course, the theoretical world
may differ from the real world and we'll be interested in studying the
relationship between two world. For instance let us consider a coin toss. In the
theoretical world we \emph{expect} to get $5$ heads out of ten tosses, yet if we
were to conduct a little experiment we may end up \emph{getting} $6$ heads out
of ten tosses.

In some cases (like tossing a bottle cap) we may not even have a theoretical
model, so the question arises that what we can conclude in this case? 

\subsection{Probability basics}

\begin{description}
  \item[Outcomes] are possible data that our experiment may result in. The set
  of all possible outcomes is the \emph{sample space} and it's noted as $S$.
  \item[Event] is any subset of the sample space.
  \item[Probability] -- each event has it's own probability to turn true, and
  for event $A$:
  \[
  0 \leq \mbox{P}(A) \leq 1 
  \]
  For example, the probability of some outcome is $1$, as we always end up
  having some result. The probability of the comploment event (meaning that
  event $A$ is not true) is:
 \[
  \mbox{P}(\bar{A}) = 1 - \mbox{P}(A)
 \] 
\end{description}

The term probability may have multiple interpretations: applies to a theoretical
world with a theoreticaal model where the probability for an event to occur is
P$(A)$; or we can look it as a long run, meaning that if we repeat the
experiment over and over again for a long time the occurance of event $A$ is
P$(A)$ fraction of all the events; another one is the subjective one: in my
opinion the chance that the event to occur is P$(A)$.

\subsection{Probability distributions}

For a coin or bear bottle fliping we use the binomial, not ass
B$(2,\frac{1}{2})$  distribution. If the exponential of the binomial is
one, we refer to it as te Bernoulli distribution: Bernoulli$(\frac{1}{2}) =
$B$(1, \frac{1}{2})$. The rolling of a dice is a discrete uniform
distribution.

\emph{Mean} is the expected value, what it equales ,,on average'' 

\[ \mbox{mean} = \mu = \sum_{x}x\mbox{P}(x)
\]

For instance in case of a rolling dice with six side:

\[
\mbox{mean} =
1\cdot\frac{1}{6}+2\cdot\frac{1}{6}+3\cdot\frac{1}{6}+4\cdot\frac{1}{6}+5\cdot\frac{1}{6}+6\cdot\frac{1}{6}=\frac{7}{2}=3.5
\]

For flipping two coins, with $Y$ being the total number of heads: 

\[
\mbox{mean}=E(Y)= 0\cdot\frac{1}{4}+1\cdot\frac{1}{2}+2\cdot\frac{1}{4} = 1
\]

\emph{Variance} in the theoretical world measures the spread of the values from
their mean value. The formula is:

\[
\mbox{variance} = \sum_{x} (x-\mu)^2\cdot\mbox{P}(x) 
\]

So for one coing flipping:

\[ \mbox{variance} =  \left(1-\frac{1}{2}\right)^2 +
\left(0-\frac{1}{2}\right)^2 + =  \frac{1}{4} \]

The \emph{standard deviation} is: 
\[ \mbox{SD} = \sqrt{\mbox{variance}} = \sqrt{\frac{1}{4}} = \frac{1}{2}
\]

The mean is linear, so any linear combination of two random variables may be
expressed with their means:

\[ \mbox{E}(aX+bY) = a\cdot\mbox{E}(X) + b\cdot\mbox{E}(Y)\]

For variance:



\begin{align*}
 \mbox{Var}(aX) &= a^2\cdot\mbox{Var}(X) \\ 
 \mbox{Var}(aX+b) &= a^2\cdot\mbox{Var}(X) \\
  \mbox{SD}(aX) &= |a|\cdot\mbox{SD}(X) 
\end{align*}

If $X$ and $Y$ are independent:
\[ \mbox{Var}(X+Y) = \mbox{Var}(X) + \mbox{Var}(Y) 
\]

Discrete random variables has a finit number of possible outcomes, and thus may
be enumerated in a form of a list. Continous random varaibles can take any value
inside an interval. An instance of this is the unifrom variable, for instance on
the interval from zero to one, meaning it's equally likeley to be any number
inside this interval. 

So for example P$(0 \leq X \leq 1) = 1$, and P$(0 \leq X \leq \frac{1}{3}) =
\frac{1}{3}$; generally speaking P$(a \leq X \leq b) = b - a$, if $0 \leq a \leq
b \leq 1$. This does mean that if $a=b$ the probability is zero, so it's easier
to think of continous probability as the area under the graph of the density
function. It's important that for any density function the total area of the
entire graph to be equal with $1$.

Uniform distributions are of a form of square function, however other functions
exist to like the exponential$(1)$ function has the form of: 

\[
 f(x) =
  \begin{cases}
   e^{-x} & \text{if } x > 0 \\
   0       & \text{if } x \leq 0
  \end{cases}
\]

The standard normal (Gaussian) distribution (bell--curve): 

\[ f(x)=\frac{1}{\sqrt{2\pi}} \cdot e ^{-\frac{x^2}{2}}\]

New bell--curves may be constructed by shifting the upper with $\mu$ (making it
the new center point) and stretching it by a factor of $\sigma$, and is noted as
Normal$(\mu, \sigma^2)$. If we have a random variable $X$ from this newly
constructed normal distribution we may transform it into a standard normal
distribution by: 

\[ 
Z = \frac{X - \mu}{\sigma} , \mbox{~ where $Z ${\raise.17ex\hbox{$\scriptstyle\mathtt{\sim}$}} Normal} (0,1). 
\]

For expected values and standard deviation now we use integals instead of sums,
for example the expected value of the uniform distribution between $0$ and $1$
is:

\[ \mbox{E}(X) = \int_{0}^{1}x\mbox{d}x = \frac{1}{2}  \]

with it's variance:

\[ \mbox{Var}(X) = \int_{0}^{1}\left(x - \frac{1}{2}\right)^2\mbox{d}x =
\frac{1}{12}
\]

For the exponential distribution it's expected value is:

\[ \mbox{E}(X) = \int_{0}^{\infty}x\cdot e^{-x}\mbox{d}x = 1  \]

with it's variance:

\[ \mbox{Var}(X) = \int_{0}^{\infty}\left(x - 1\right)^2\cdot e^{-x}\mbox{d}x =
1 \]

For the $X ${\raise.17ex\hbox{$\scriptstyle\mathtt{\sim}$}} Normal $(0,1)$: 

\[ \mbox{E}(X)  = \int_{-\infty}^{\infty}x \cdot \frac{1}{\sqrt{2\pi}}
\cdot e^{-\frac{x^2}{2}} \mbox{d}x = 0 
\]

\[ \mbox{Var}(X)  = \int_{-\infty}^{\infty} \left(x - 0\right)^2 \cdot
\frac{1}{\sqrt{2\pi}} \cdot e^{-\frac{x^2}{2}}\mbox{d}x = 1 
\]

And in case of $Y ${\raise.17ex\hbox{$\scriptstyle\mathtt{\sim}$}} Normal
$(\mu,\sigma^2)$ the mean is $\mu$, variance of $\sigma^2$ and standard
deviation of $\sigma$.

\subsection{Long running averages} 

What happens if you repeat an experiment lots of times and you look at the
average value you get. For example if you start flipping a coin a lot of times
and you look at the fraction of times you get head, you expect that the more you
do it, the more this comes closer to half. If you were to draw the probabilities
of the coin flip on a graph, you'd observe that the shape starts to resemble the
density function for the normal distribution. The same is the case for dice
rolling at looking the average of the rolled numbers.

\emph{The Law of Large Numbers} states that if an experiment is repeated over
and over, then the average result will converge to the experiment's expected value.

\emph{The Central Limit Theorem} states that if an experiment is repeated over
and over, then the probabilities for the average result will converge to a
Normal--distribution.

Suppose that an experiment is repeated over and over, with outcomes: $X_1, X_2,
$ $~\ldots$ and suppose each mean is E$(X_i) = m$, and each variance is is
Var$(X_i)=v$. Now let $\bar{X} = \frac{X_1 + X_2 +\ldots + X_n}{n}$ be the
average outcome. In this case we can say that:

\[ \mbox{E}(\bar{X}) = \frac{\mbox{E}(X_1) + \mbox{E}(X_2) + \ldots +
\mbox{E}(X_n)}{n} = \frac{nm}{n} = m,\]

\[ \mbox{Var}(\bar{X}) = \frac{\mbox{Var}(X_1) + \mbox{Var}(X_2) + \ldots +
\mbox{Var}(X_n)}{n} = \frac{nv}{n^2} = \frac{v}{n},\]

so we can conclude as n rises to infinity the variance (uncerteinty) becomes
smaller and smaller, tending to zero; with this the standard deviation too.

The centreal limit theorem also is responsible for the emparical rule; the
percentages are true for the graph of the normal distribution. In conclusion we
can say that all that is some kind of average, or is made up of lots and lots of
small contributions usually has a normal distribution behind it.

\subsection{Sampling distribution}

From the real world we collect samples, which can be considered as part of the
theoretical worlds population. We have scientific and statistical models in the
theoretical world which have parameters of features that we do not know. We use
the science of statistics to estimate them.

Let there be $p$ our parameter of interest, what we are observing, and let it
denote the number of heads of a coin flip sequence. From the real world (data)
we get some result. With this we can esitmate the parameter as:

 \[ \hat{p} = \frac{\mbox{numbers of heads observed}}{\mbox{number of coin
 flips}}\].
 
 The observed value of an estimator varies from sample of data to sample of
 data. The variability of this is called the \emph{sampling variability}. The
 probability distributions of the possible value of an estimator is its sampling
 distribution.
 
 A statistic used to estimate a paramters is \emph{unbiased} if the expected
 value of its sampling distribution is equal to the value of the parameter being
 estimated (the sampling probability is the same as the theoretical
 probability). How close the distribution of the sampling gets to the parameter
 dependes on the varince of the sampling distribution. This decreases with the
 number of samples, so the more data we have the closer we get to the
 theoretical word.
 
 Following the central limit theorem for large $n$ the sampling distribution for
 a Bernoulli event (happens or not, with probability $p$) is N$\left( p,
 \frac{p(1-p)}{n}\right)$ (sampling distribution of $\hat{p}$).
 
 If we take a normal distribution as a sample distribution with a normal
 distribution theoretical model we can calculate the expected value of the
 theoretical model as the average of the sample we have, and the standard
 deviation of the theoretical model decreases with the number of data compared
 to the standard deviation of the sampling distribution
 ($\frac{\sigma}{\sqrt{n}}$). The sampling distribution of $\bar{X}$ is N$\left(
 \mu, \frac{\sigma^2}{n}\right)$.
 
 For calculating the variance dividing with $n-1$ is important to make the
 estimator unbiased, while dividing with $n$ will result in a bias estimator.
 
